

% The (simplified) syntax of Creol is given in Fig.~\ref{fig::creolSyntax}.
% Here we introduce the basic concepts of Creol.
%
%
% \newcommand{\trans}[1]{\,{\stackrel{{#1}}{\longrightarrow}}\,}
% \newcommand{\gtrans}[2]{\,{\stackrel{{#2}}{\longrightarrow_{#1}}}\,}
% \newcommand{\transtwo}[1]{\, { \stackrel {{  \stackrel{{#1}}{\longrightarrow} }} {{#1}} }   \,}
% \newcommand{\M}{\mathcal{M}}
% \newcommand{\Nat}{\mathbb N}
% \newcommand{\Act}{\Sigma}
% \newcommand{\id}[1]{\mathit{#1}}
% 
% \begin{figure}[t]
% \begin{center}
% \begin{tabular}{l@{~~~}l}
% \begin{math}
% \begin{array}{lcl}
% \id{IF} & ::= & \id{\bf interface}\ N
% %\{[\{N\}_,^+]\}^?
% \{(\id{Par})\}^?
% %\\~&~&
% \{\id{\bf inherits}\ \id{Inh}\}^?  \\
% ~   & ~ &  \id{\bf begin}\ \{\id{\bf with}\ \id{N}\ \id{Msig}^+\}^?\ \id{\bf end} \\
% %\id{Type} & ::= & N\{[\{\id{Type}\}_,^+]\}^? \\
% \id{Inh}  & ::= & \{\id{N}\{(E)\}^?\}_,^+ \\
% \id{Par} & ::= & \{\{v\}_,^+ :\id{N}\}_,^+ \\
% \id{Msig} & ::= & \id{\bf op}\ N \{(\{\id{\bf in}\ \id{Par}\}^?\ \{\id{\bf out}\ \id{Par}\}^?)\}^? \\
% \id{CL} & ::= & \id{\bf class}\ N
% %\{[\{N\}_,^+]\}^?\
% \{(\id{Par})\}^?
% \\
% ~&~&
%  \{\id{\bf contracts}\ \id{Inh}\}^? \  \{\id{\bf inherits}\ \id{Inh}\}^?  \\
% ~ & ~ & \id{\bf begin} \ \id{Vdcl}^? \{\{\id{\bf with}\ \id{N}\}^?\ \id{Mtd}\}^*\ \id{\bf end} \\
% \id{Vdcl} & ::= & \id{\bf var}\ \{\{v\}_,^+: \id{N}\{=e\}^?\}_,^+ \\
% \end{array}
% \end{math}
% &
% \begin{math}
% \begin{array}{l}
% \id{Mtd} ::= \{\id{Msig} ==\{\id{Vdcl};\}^? \ S\}^+ \\
% g\ ::=\ b\ |\ t?\ |\ \lnot g\ |\ g \land g \\
% p\ ::= \ x.m \ |\ m \\
% S\ ::=\ \epsilon\ | \ s;S \\
% s\ ::=\ (S)\ |\ V := E\ |\ \id{\bf skip} \\
% ~~~~~ |\ v := \id{\bf new}\ N(E)\ | \ !p(E) \\
% ~~~~~ | \ t!p(E) \ | \ t?(V)\ | \ p(E;V) \\
% ~~~~~ | \ \id{\bf if}\ b\ \id{\bf then}\ S\ \id{\bf else}\ S\ \id{\bf end}\\
% ~~~~~ | \ \id{\bf await}\ g \ | \ \id{\bf await}\  t?(V)\\
% ~~~~~ | \ \id{\bf await}\  p(E;V)\ |\ \id{\bf release} \\
% \end{array}
% \end{math}
% \end{tabular}
% \end{center}
% \caption{BNF grammar for Creol. Curly
% brackets are used as meta parenthesis, superscript ? for optional
% parts, superscript * for repetition zero or more times, whereas
% $\{. . .\}_,^+$  denotes repetition one or more times with �,� as delimiter. Identifiers N denote interface, class, type, or method names.
% %The language syntax for method definitions, with typical
% %terms for each category.
% Capitalized terms such as $E$, $V$, and $S$,
% denote lists of the syntactic categories of the corresponding lower-case terms~\cite{johnsen07sosym,JohnsenOY06}.}\label{fig::creolSyntax}
% \end{figure}


Creol~\cite{creol:broch_owe} (Concurrent Reflective Object-oriented Language) is
an object oriented modeling language designed for specifying distributed
systems. A Creol object implicitly has a dedicated processor and thus
encapsulates an execution thread. Different objects communicate only by
asynchronous method calls, i.e., similar to message passing in Actor models
\cite{actors:agha};  however in Creol, the caller can poll or wait for return
values or termination of the called method. This can also be used to simulate
synchronous method calls. % we can say: like RPC We use an exclusive resource
%object as a running example, shown in Listing \ref{lst:exres}.

When an object is created, its local state is initialized by executing the \footnotesizett{init} method.
Then the object starts its active behavior by executing its
\footnotesizett{run} method if defined. % (e.g., in class \footnotesizett{User}). 
When receiving a method call a new process is created to execute the method. 
Methods can have processor release points which define interleaving points explicitly.
When a process is executing, it is not interrupted until it finishes or reaches a release point.
Release points can be conditional:
%, e.g.,  \footnotesizett{await mutex.request()} in Listing \ref{lst:exres} checks for termination of \footnotesizett{request}. 
if the guard at a release point evaluates to true, the
process keeps the control, otherwise, it releases the processor and becomes
disabled as long as the guard is not true. Whenever the processor is free, an
enabled process is {\em nondeterministically} selected for execution, i.e.,
scheduling is left unspecified in Creol in favor of more abstract modeling.


%%%%%%%%%%%%%%%%%%%%%%%%%%%%%%%%%%%%%%%%%%%%%%%%%%%%%%%%%%%%%%%%%%%%%%%%%%%%%%%%%%%%%%%%%%%%%

% \subsection{Creol}
% Creol \cite{creol:web} stands for Concurrent Reflective Object-oriented
% Language. Creol is a language that provides programming constructs and
% reasoning control in the context of open distributed systems
%  taking an object-oriented approach
% \cite{creol:web,creol:broch_owe,mpd:andrews}.
% 
% Johnson and Owe \cite{creol:broch_owe} focus on synchronization of distributed systems
% and the challenges. They argue that it is difficult to combine active and
% passive behavior in concurrent objects; so, they propose Creol as an
% object-oriented language for the solution mainly taking advantage of
% \textit{processor release points} and a notion of \textit{asynchronous method
% call}. Creol is designed to unite object orientation and distribution and helps
% to dynamically change between active and reactive behavior.
% 
% %According to \cite{creol:broch_owe,mpd:andrews}, the basic interaction models
% %for concurrent processors are \textsl{shared variables}, \textsl{remote method
% %calls}, and \textsl{message passing}. Shared variables do not generalize well
% %with distributed environments. In remote method call (RMI), the caller is
% %blocked until the response is received. In contrast, as message passing does not
% %transfer control between concurrent objects is not blocking. Message passing can
% %either be \textit{synchronous} or \textit{asynchronous}. In synchronous mode,
% %both sender and receiver should be ready to take part in the interaction,
% %however, in asynchronous mode, a message can be sent at any time regardless of
% %when the receiver accepts the message.
% 
% Creol's message passing is analogous to Actor Model \cite{actors:agha}. However,
% they \cite{creol:broch_owe} believe that actors do not distinguish replies from
% invocations, so capturing method calls with actors quickly becomes unmanageable.
% In Creol, method calls are taken as the communication primitive for concurrent
% objects. In Creol, an object encapsulates an execution thread.
% %Thus, some data types are not considered objects inside an
% %object such as basic data types. Each object has a unique identity (name)
% %through which the communication with other objects occurs.  Object names are
% %exchanged with one another. 
% 
% Creol is a language by \textsl{interface}. Every object implements an interface
% in the language \cite{creol:broch_owe,creol_overview:blanchette}. Creol
% introduces the notion of \textsl{cointerface} to be used when the callee may
% need to access to caller object. In this case, a \texttt{with} clause is used to
% specify and restrict access to different objects supporting different
% interfaces. 
% 
% %Creol provides inheritance a bit different of what is generally perceived. A
% %\textsl{class} can \texttt{implement} an interface, \texttt{contract} an
% %interface, or \texttt{inherit} a class. The difference lies in the semantics
% %that is passed down to the hierarchy. In case of contracting, the signature of
% %super interfaces are also present in the semantics of the object of the class.
% %However, in the case of inheriting, the semantics of the super class is not
% %passed down \cite{creol_overview:blanchette}.
% 
% In Creol, an asynchronous message can be sent at any time
% \cite{creol:broch_owe}. Also, \textit{method overtaking} is allowed; i.e. if
% some methods offered by some object are invoked in the same order, it is up to
% the object to reorder the \textit{method instances} in execution. When the
% caller needs the return values from the callee, it may ask for it in which case
% it may be suspended based on the processor release points in Creol. Moreover, a
% message can be synchronous causing the caller to suspend till the reply is
% ready. In Creol, a callee does \textit{not} distinguishes synchronous and
% asynchronous invocation.
% 
% Creol introduces the notion of processor release point
% \cite{creol:broch_owe,creol_overview:blanchette} to provide an explicit
% mechanism to release or acquire processing unit during execution. Essentially,
% clause \texttt{await g} releases processor if guard expression \texttt{g}
% evaluates to false and grabs it again later when \texttt{g} is true. Also,
% \texttt{wait} explicitly releases processor. Creol's asynchronous message
% passing takes advantage of such processor release point model.
% 

