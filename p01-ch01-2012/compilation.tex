******************************************************************

In this section, we briefly discuss how different constructs in Creol and Java
co-relate to one another in the compilation process.
Figure \ref{fig:creol_object} depicts a general concurrent object in Creol that is transformed 
to an equivalent Java object with a single thread of execution on a processor.

\subsubsection{Active Objects to Task Objects}
In the transformation process, one of the most important questions is that: how
Creol's concurrency model maps to concurrent Java?

Based on Creol's notion of active objects, each object conceptually possesses
one processing core in Creol. It means that each object is a thread of execution
and sends messages to other objects. Each Creol object is mapped to one Java
object and one concurrent task. One major advantage of this approach is that
since all the object's belongings stay on the same core, the problems of
synchronization and shared state issues will not occur.

Every object maintains a storage of the messages that is used to receive and
store messages sent from other objects (mailbox). At some time $t_i$, a message
is received in the object's mailbox. However, $t_i$ is not necessarily the time
that the message is processed as the same as what happens in Creol. As some
other $t_e$, the objects looks in the mailbox to select a message to process (it
can be the case that $t_i = t_e$). 

In the ``Active Segment'' in Figure \ref{fig:creol_object}, when a method
invocation is received in the object, it carries information on priority
specification possibly from both the client and the server. This information is
used by a ``priority manager'' component in the object to assign the message a
final priority value based on which the message is stored in the mailbox of the object.
On the other hand, each object consists of a ``scheduler'' responsible to select
one message at a time for execution. The scheduler component may use predefined
policies specified by the user programmer how to select one message at a time.
It may also be developed and injected by the user programmer to provide further
extensibility and customization. When a message is selected, the object executes
the requested method.

The Java object has access to an instance of
{\footnotesize\texttt{ExecutorService}} to encapsulate a single thread of
execution on a processor. Each method invocation is transformed to a variation
of  {\footnotesize\texttt{Callable}} and stored in the object's mailbox. The
method invocations return instances of {\footnotesize\texttt{Future}} to hold
the eventual result of the method.

\subsubsection{Asynchronous Method Calls}
With an asynchronous method call in Creol, the object continues with the
next instruction after the call and the response from the invoked method will be
stored in a labeled variable that can be later checked or fetched. Creol can
also model a synchronous method with an asynchronous method call.
 
Similarly, concurrent Java provides an implementation of
{\footnotesize\texttt{Future}}\footnote{\texttt{java.util.concurrent.Future}}
computation. It encapsulates a value that will be determined sometime later in
the time of execution or cancelled during execution. Having a future variable in
Java, a Creol asynchronous method call can be mapped to an instance of
{\footnotesize\texttt{FutureTask}}\footnote{\texttt{java.util.concurrent.FutureTask}}.

When a method invocation request is received in the object, an instance of
{\footnotesize\texttt{FutureTask}} is created to encapsulate the original method
invocation. As the same time, an instance of {\footnotesize\texttt{Future}} is
returned as the ``future'' result of this method invocation. The actual result
will be eventually available when the message is selected for execution. On the
other hand, the incoming message carries priority scheduling specification. So,
the priority manager of the object uses such information to map different levels
of priorities possibly specified for the method invocation to a final priority
value. This value is assigned to the message being stored in the mailbox and is
used by the scheduler. When the scheduler of the object selects one method
invocation from the mailbox, it uses the mapped final priority values.

\subsubsection{Creol Processor Release Points}
One of the most important primitives in Creol language is the ability to
\textit{manually} release a processor. Creol allows the programmer to decide
through syntax whether to yield the processor or continue the process. On the
other hand, using concurrent Java provides the same functionality using
{\footnotesize\texttt{Condition}}s\footnote{\texttt{java.util.concurrent.locks.Condition}}.
When a program in Java takes advantage of a condition object, it is actually
controlling the enclosing executing thread. The condition mechanism is in
analogy with the older Java's Object {\footnotesize\texttt{wait}} and
{\footnotesize\texttt{notify}} mechanism. The transformation process utilizes
this feature in Java to provide the explicit {\footnotesize\texttt{await}}
syntax in Creol.


