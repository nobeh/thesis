% !TEX root = thesis.tex
%
\pdfbookmark[0]{Abstract}{Abstract}
\chapter*{Abstract}
\label{sec:abstract}
\vspace*{-10mm}

Object orientation provides principles for abstraction and encapsulation.
Abstraction is accomplished by high-level concepts of interfaces, classes, and objects.
Many object-oriented languages have followed the blocking and synchronous model of messaging. 
\emph{Asynchronous} message passing is one of the fundamental elements of the actor model.
Unlike in object orientation, a message is not bound to any interface in the actor model.

Ongoing research focuses on combining object-oriented programming with the actor model and concurrency.
In a concurrent setting with objects, multi-threading is a common approach to provide an object-oriented concurrency model.
Integration of actor model and object orientation leads to the use of asynchronous method calls.
However, neither of object-orientation and actor model support explicit control of pre-emption for messages or asynchronous method calls.
%In contrast, a message in actor model cannot be pre-empted while being processed.
In comparison to this rigid ``run to completion''-execution style, coroutines are more flexible: they allow multiple entry points and can be used for collaborative pre-emption between tasks by giving up control with explicit yield or suspend instructions.
Coroutines are not originally established in the object-oriented paradigm.
Furthermore, interactions in both coroutines and multi-threading are blocking and synchronous.
The main challenge is to generate production code from an actor-based language which supports asynchronous method calls and coroutine-style execution.

This thesis contributes to the intersection of object orientation, the actor model, and concurrency.
We choose Java as the main target programming language and as one of the mainstream 
object-oriented languages. 
We formalize a subset of Java and its concurrency API.
We create an abstract mapping from an actor-based modeling language, ABS, to the programming semantics of concurrent Java. 
The mapping ensures a correct translation from a model to its equivalent production-ready code.
We provide the formal semantics of the mapping and runtime properties of 
the concurrency layer including deadlines and scheduling policies.
We provide an implementation of the ABS concurrency layer as a Java API library and framework using Java~8 features.
The implementation is generic and extensible as it can be used either as a standalone library or as part of a pluggable code generator.
We design and implement a runtime monitoring framework, JMSeq, to verify the
correct ordering of execution of methods in possibly interleaved threads through code annotations in JVM. 
In addition, we design a large-scale monitoring system as a real-world 
application; the monitoring system is built with ABS concurrent objects 
and formal semantics that leverages schedulability analysis to verify correctness of the monitors.

\vspace*{20mm}

{\usekomafont{chapter}Samenvatting}\label{sec:abstract-diff} \\
%Object orientation provides principles for abstraction and encapsulation.
Objectgeoriënteerd programmeren biedt ondersteuning voor abstractie en encapsulatie.
%Abstraction is accomplished by high-level concepts of interfaces, classes, and objects.
Abstractie wordt bereikt door de introductie van high-level concepten als interfaces,
klassen en objecten.
%Many object-oriented languages have followed the blocking and synchronous model of messaging. 
Veel objectgeoriënteerde talen gebruiken blokkerende, synchrone communicatie voor het zenden van berichten (messages) tussen objecten.
%\emph{Asynchronous} message passing is one of the fundamental elements of the actor model.
\emph{Asynchrone} communicatie (message passing) vormt een fundamenteel element van het zogenaamde actor model.
%Unlike in object orientation, a message is not bound to any interface in the actor model.
Anders dan in objectoriëntatie is een message niet gebonden aan of beperkt tot een specifieke interface.

%Ongoing research focuses on combining object-oriented programming with the actor model and concurrency.
Huidig onderzoek focust op het combineren van objectoriëntatie met het actor model en concurrency (meerdere taken kunnen tegelijk actief zijn).
%In a concurrent setting with objects, multi-threading is a common approach to provide an object-oriented concurrency model.
Voor het uitbreiden van objectoriëntatie naar een concurrent omgeving is multi-threading een veelgebruikte aanpak.
%Integration of actor model and object orientation leads to the use of asynchronous method calls.
Integratie van objectoriëntatie met het actor model leidt tot het gebruik van asynchrone methode aanroepen.
%However, neither of object-orientation and actor model support explicit control of pre-emption for messages or asynchronous method calls.
Echter, nog objectoriëntatie, nog het actor model biedt ondersteuning voor expliciete controle voor het tijdelijk onderbreken (pre-emption) van de verwerking van berichten of asynchrone methode aanroepen.
%In contrast, a message in actor model cannot be pre-empted while being processed.
%REMOVED THIS SENTENCE
%In comparison to this rigid ``run to completion''-execution style, coroutines are more flexible: they allow multiple entry points and can be used for collaborative pre-emption between tasks by giving up control with explicit yield or suspend instructions.
In vergelijking met deze starre ``run to completion''-executie stijl van programma's zijn coroutines flexibeler: er zijn meerdere ``entry points'' voor de executie, en taken coöpereren voor controle over de processor, door tijdelijk controle op te geven met expliciete yield of suspend instructies.
%Coroutines are not originally established in the object-oriented paradigm.
Coroutines zijn echter niet standaard in objectoriëntatie,
%Furthermore, interactions in both coroutines and multi-threading are blocking and synchronous.
en interacties met zowel coroutines als multi-threading zijn traditioneel synchroon en blokkerend.
%The main challenge is to generate production code from an actor-based language which supports asynchronous method calls and coroutine-style execution.
De belangrijkste uitdaging is de generatie van productie code uit een actor gebaseerde taal die ondersteuning biedt voor asynchrone methode aanroepen en een flexibele co-routine stijl van executie.

%This thesis contributes to the intersection of object orientation, the actor model, and concurrency.
Dit proefschrift draagt bij aan de kennis over de combinatie van objectoriëntatie, het actor model en concurrency.
%We choose Java as the main target programming language and as one of the mainstream 
%object-oriented languages. 
We kiezen Java, als een volwassen en een van de meest gebruikte programmeertalen in productie omgevingen, als de target language voor code generatie vanuit een actor gebaseerde taal.
%We formalize a subset of Java and its concurrency API.
We formaliseren een deel van Java en zijn concurrency library (API).
%We create an abstract mapping from an actor-based modeling language, ABS, to the programming semantics of concurrent Java. 
Vervolgens creëren we een abstracte relatie (een functie) van de actor gebaseerde taal ABS naar concurrent Java.
%The mapping ensures a correct translation from a model to its equivalent production-ready code.
Deze abstracte functie garandeert een correcte vertaling van een abstract ABS model naar executeerbare productie code.
%We provide the formal semantics of the mapping and runtime properties of 
%the concurrency layer including deadlines and scheduling policies.
We geven de formele semantiek van de functie en de eigenschappen van de concurrency
laag tijdens executie, inclusief deadlines en scheduling policies.
%We provide an implementation of the ABS concurrency layer as a Java API library and framework using Java~8 features.
Daarna implementeren we de concurrency laag van ABS als een Java API library, met behulp van Java~8.
%The implementation is generic and extensible as it can be used either as a standalone library or as part of a pluggable code generator.
Deze implementatie is generiek en uitbreidbaar, en kan zowel gebruikt worden als
standalone library, of als een inplugbare code generator.
%We design and implement a runtime monitoring framework, JMSeq, to verify the
%correct ordering of execution of methods in possibly interleaved threads through code annotations in JVM. 
We ontwerpen en implementeren een monitoring framework genaamd JMSeq waarmee
de correcte ordening van methode aanroepen, gespecificeerd met code annotaties, geverifieerd kan worden in een multi-threaded omgeving.
%In addition, we design a large-scale monitoring system as a real-world 
%application; the monitoring system is built with ABS concurrent objects 
%and formal semantics that leverages schedulability analysis to verify correctness of the monitors.
Tevens ontwerpen we een grootschalig real-world monitoring systeem met
ABS concurrent objecten en een formele semantiek. We gebruiken state-of-the-art technologie voor schedulability analyse om de correctheid van de monitors te verifiëren.