% !TEX root = thesis.tex
%
\pdfbookmark[0]{Abstract}{Abstract}
\chapter*{Abstract}
\label{sec:abstract}
\vspace*{-10mm}

Object orientation provides principles for abstraction and encapsulation.
Abstraction is accomplished by high-level concepts of interfaces, classes, and objects (class instances).
Many object-oriented languages have followed the blocking and synchronous model of messaging. 
One of the fundamental elements of the actor model is \emph{asynchronous} message passing.
However, a message is not bound to any interface in the actor model.
It has been an ongoing research to combine object-oriented programming with the actor model.
Multicore and distributed computing raised a new challenge: combining object orientation, the actor model, and \emph{concurrency}.
In a concurrent setting, coroutines enable interactions with collaborative pre-emption.
Coroutines are not originally established in the object-oriented paradigm.
In a concurrent setting with objects, multi-threading has been a common approach to provide an object-oriented concurrency model.
Furthermore, interactions in both coroutines and multi-threading are blocking and synchronous.
In contrast, the actor model relies on asynchronous communication.
While a message is processed, an actor cannot be pre-empted or intentionally yield to allow other actors in the system to make progress.
Integration of actor model and object orientation leads to the use of asynchronous method calls.

This thesis contributes to the intersection of object orientation, actor model, and concurrency.
We choose Java as the main programming language and as one of the mainstream 
object-oriented languages. 
We formalize a subset of Java and its concurrency API to 
facilitate formal verification and reasoning about it.
We create an abstract mapping from a concurrent-object modeling language, 
ABS, to the programming semantics of concurrent Java. 
We provide the formal semantics of the mapping and runtime properties of 
the concurrency layer including deadlines and scheduling policies.
We provide an implementation of the ABS concurrency layer as a Java API library 
and framework utilizing the latest language additions 
in Java~8.
We design and implement a runtime monitoring framework, JMSeq, to verify the
correct ordering of execution of methods through code annotations in JVM. 
In addition, we design a large-scale monitoring system as a real-world 
application; the monitoring system is built with ABS concurrent objects 
and formal semantics that leverages schedulability 
analysis to verify correctness of the monitors.

\vspace*{20mm}

{\usekomafont{chapter}Samenvatting}\label{sec:abstract-diff} \\

\TODO{Translate to Dutch. Kindly by Stijn.}
