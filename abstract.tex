% !TEX root = thesis.tex
%
\pdfbookmark[0]{Abstract}{Abstract}
\chapter*{Abstract}
\label{sec:abstract}
\vspace*{-10mm}

Object orientation provides principles for abstraction and encapsulation.
Abstraction is accomplished by high-level concepts of interfaces, classes, and objects.
Many object-oriented languages have followed the blocking and synchronous model of messaging. 
One of the fundamental elements of the actor model is \emph{asynchronous} message passing.
Unlike object orientation, a message is not bound to any interface in the actor model.
Ongoing research focus to combine object-oriented programming with the actor model and concurrency.
In a concurrent setting with objects, multi-threading is a common approach to provide an object-oriented concurrency model.
In contrast, a message in actor model cannot be pre-empted while being processed.
Integration of actor model and object orientation leads to the use of asynchronous method calls.
% Multicore and distributed computing raised a new challenge: combining object orientation, the actor model, and \emph{concurrency}.
Coroutines can be used for collaborative pre-emption.
Coroutines are not originally established in the object-oriented paradigm.
Furthermore, interactions in both coroutines and multi-threading are blocking and synchronous.
The main challenge is to generate production code from an actor-based language which supports asynchronous method calls and coroutine-style execution.

This thesis contributes to the intersection of object orientation, actor model, and concurrency.
We choose Java as the main programming language and as one of the mainstream 
object-oriented languages. 
We formalize a subset of Java and its concurrency API.
We create an abstract mapping from a actor modeling language, ABS, to the programming semantics of concurrent Java. 
The mapping ensures a correct translation from a model to its equivalent production-ready code.
We provide the formal semantics of the mapping and runtime properties of 
the concurrency layer including deadlines and scheduling policies.
We provide an implementation of the ABS concurrency layer as a Java API library and framework using Java~8 features.
The implementation is generic and extensible as it can be used either as a standalone library or as part of a pluggable code generator.
We design and implement a runtime monitoring framework, JMSeq, to verify the
correct ordering of execution of methods in possibly interleaved threads through code annotations in JVM. 
In addition, we design a large-scale monitoring system as a real-world 
application; the monitoring system is built with ABS concurrent objects 
and formal semantics that leverages schedulability analysis to verify correctness of the monitors.

\vspace*{20mm}

{\usekomafont{chapter}Samenvatting}\label{sec:abstract-diff} \\

\TODO{Translate to Dutch. Kindly by Stijn.}
