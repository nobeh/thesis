% -*- program: pdflatex -*-

\documentclass[a5paper,10pt]{article}
\usepackage[scale=0.9,centering]{geometry}
\usepackage[utf8]{inputenc}
\usepackage[T1]{fontenc}
\usepackage{charter}

\pagenumbering{gobble}
 
\begin{document}
 
\begin{center}
{\Large \uppercase{Stellingen}}\\
Propositions belonging to the Ph.D. dissertation:\\
{\large Actors at Work}\\
by \textsl{Behrooz Nobakht}\\[1.5em]
\end{center}

\begin{enumerate}
\item Application-level priorities is one of the cross-functional requirements
of concurrent systems as important as functional requirements. (Chapter 2)

\item Co-operative scheduling of application-level priorities 
is resourceful and costly in thread-based programming languages. (Chapter 2)

\item Real-time programming with deadlines and timeouts requires 
fine granularity and flexibility of the programming language. (Chapter 3)

\item Separation of invocation from execution of asynchronous messages
drastically minimizes resourcefulness of co-operative scheduling
in a multi-threaded runtime. (Chapter 4)

\item Java 8 features enable to create a more efficient abstraction for 
asynchronous messages. (Chapter 4)

\item Runtime verification of multi-thread Java applications through
non-intrusive code annotations processing improves correctness and de-coupling in
comparison to bytecode instrumentation. (Chapter 5)

\item Actors are a natural fit for a distributed monitoring model
based on observation and reaction to guarantee composable 
multi-objective service characteristics. (Chapter 6)

\item Bridging modelling and programming for the purpose of verification 
and reasoning is a necessity for concurrent and distributed programming.

\item Explicit transfer of control in co-operative scheduling in ABS makes
programs easier to understand and reason about compared to the implicit
behavior in multi-threaded models.

\item Supporting a hybrid thread mapping model in object-oriented languages
such as Java is a desirable but currently unrealized feature because of implementation
complexity.

\item The orthogonal properties of object orientation, co-operative scheduling,
and actor model has given rise to numerous challenges to devise a new 
programming model at their intersection.

\item The increasing importance of concurrent and distributed programming,
similar to the shift of structured programming to object orientation,
requires more of a mentality change rather than languages and tools.

\item Concurrency could be hard even for a computer despite the presence of an 
experienced programmer.
\end{enumerate} 
 
\end{document}