% -*- root: paper-podc.tex -*-

\section{SDL Fredhopper Cloud Services} % (fold)
\label{ch04:sec:fredhopper_example}

In this section, we introduce a running example in the context of SDL Fredhopper. 
We use the example in different parts of the paper and also in the experiments.

SDL Fredhopper develops the \frhcsp to offer search and targeting facilities on a large product database to e-Commerce companies as services (SaaS) over the cloud computing infrastructure (IaaS).  
\frhcsp provides several SaaS offerings on the cloud. 
These services are exposed at endpoints. 
In practice these endpoints typically are implemented to accept connections over HTTP.
For example, one of the services offered by these endpoints is the Fredhopper Query API, which allows users to query over their product catalog via full text search\footnote{\url{http://en.wikipedia.org/wiki/Full_text_search}} and faceted navigation\footnote{\url{http://en.wikipedia.org/wiki/Faceted_navigation}}.

A customer of SDL Fredhopper using Query API owns a \emph{single} HTTP endpoint to use for search and other operations.
However, internally, a number of resources (virtual machines) are used to deliver Query API for the customer.
The %set of
resources used for a customer are managed by a load balancer.
% to provide a single HTTP endpoint.
In this model of deployment, each resource is launched to serve \emph{one} instance of Query API; i.e. resources are \emph{not} shared among customers.

When a customer signs a contract with SDL Fredhopper, there is a clause in the contract that describes the minimal QoS levels of the
%customer regarding the
Query API.
For example, we have a notion of query per second (\qps) that defines the number of completed queries per second for a customer. 
An agreement is a bound on the expected \qps and forms
%which becomes
the basis of many decisions (technical or legal) thereafter. 
The agreement is used by the operations team to set up an environment for the customer which includes the necessary resources described above.
The agreement is additionally used by the support team to manage communications with the customer during the lifetime of the service for the customer.
% 

Maintaining the services for more than 250 customers on more than 1000 servers is not an easy operation task
\footnote{Figures are indication of complexity and scale. Detailed confidential information may be shared upon official request.}.
Thus, to ensure the agreements in a customer's contract:
\begin{itemize}
\item The operation team maintains a monitoring platform to get notifications on the current metrics.
\item The operation team performs \emph{manual} intervention to ensure that sufficient resources are available for a customer (launching or terminating).
\item The monitoring platform depends on \emph{human} reaction.
\item The cost that is spent for a customer on the basis of safety can be \emph{optimized}.
\end{itemize}

In this paper, we use the notion of \qps as an example in the concepts that are presented in this research. 
We use the example here to demonstrate how the model that is proposed in this research can address the issues above and alleviate the \emph{manual} work with \emph{automation}. 
The manual life cycle depends on the domain-specific and contextual knowledge of the operations team for every customer service that is maintained in the deployment environment.
This is labor-intensive as the operations team stands by $24\times7$.
In such a manual approach, the business is forced to over-spend to ensure service level agreements for customers. 
