% -*- root: paper-esocc.tex -*-

% Section: Related Work

\section{Related Work}
\label{sec:relatedwork}

Vast research work present different aspects of runtime monitoring.
We focus on those that present a line of research for distributed deployment of services.

%MONINA~\cite{inzinger:monitoring,inzinger2013specification} is a DSL language based on which a generic monitoring architecture is designed.
%The architecture and the language use a rule engine based on facts.
%To use MONINA, it is required to model system parts as ``component'' offered by the DSL language.
%We build our model using timed automata~\cite{alur:1994:timedautomata} with a mapping to ABS~\cite{johnsen2012abs}, a concurrent object-oriented modelling language which is essential for schedulability analysis~\cite{fersman2007task} of real-time and distributed systems.
%MONINA offers a solution based on facts and ``derived'' facts used by a rule engine to emit corrective events.
%The DSL language in MONINA introduces \emph{cost} and \emph{capacity}.
%In contrast, we introduce a generic service metric function to allow for generic definition of a metric from the service's standpoint. 
%In our model, we also study similar metrics such as cost.
%To deploy MONINA on the cloud, the authors recommend using the notion of \emph{adaptation in cycles}.
%It is similar to \emph{monitoring window} that we define further in the paper.
%However, we formally prove how the selection of the monitoring window is key to the evolution of the service.
MONINA~\cite{inzinger:monitoring} is a DSL with a monitoring architecture
which supports certain mathematical optimization techniques.
A prototype implementation is available.
%To use MONINA requires developing a MONINA component.
Accurately capturing the
behavior of an in-production legacy system coded in a
conventional language seems challenging: it requires
developing MONINA components, which
generate events at a specified fixed rate,  there are no control structures (if-else, loops), the data types that can be used in events are pre-defined, and there are no OO-features. 
We use ABS~\cite{johnsen2012abs}, an executable modeling language
that supports all of these features and offers a wide range of tool-supported analyses~\cite{BubelMH14,WongBBGGHMS15}.  The mapping from ABS to timed automata~\cite{alur:1994:timedautomata} allows to
exploit the state-of-the-art tools for timed automata, in
particular for reasoning about real-time properties (and, as we show, SLAs using
schedulability analysis~\cite{fersman2007task}).
MONINA offers \emph{two pre-defined} parameters that can be used in monitoring to adapt
the system: cost and capacity.  Our service metric function generalizes this to
\emph{arbitrary user-defined} parameters, including cost and capacity.

% Furthermore, the architecture applies a set of adaptation rules.
% The architecture uses an event-driven approach and message passing mechanisms.
% Our approach leverages the well-established formalism of timed automata~\cite{alur:1994:timedautomata} which allows for static analysis of real-time properties of a service.
% We do not impose a new language; however, we use the established formalism of timed automata~\cite{alur:1994:timedautomata}.
% We propose a generic model to be able to characterize arbitrary service properties that may come from different sources.
% The model in the paper allows for a language-agnostic implementation. 
%However, MONINA provides an architecture implementation based on Rule Engine technologies and its domain-specific language design.

Hogben and Pannetrat examine in \cite{hogben2013defavail} the challenges of defining and measuring availability to support real-world service comparison and dispute resolution through SLAs.
They show how two examples of real-world SLAs would lead one service provider to report 0\% availability while another would report 100\% for the same system state history but using a different period of time.
The transparency that the authors attempt to reach is addressed in our work by the concept of monitoring window and expectation tolerance in Section \ref{sec:modelling}.
Additionally, the authors take a continuous time approach contrasted with ours that uses discrete time advancements.
Similarly, they model the property of availability using a two-state model.

%There are related work in the domain of service level agreements (SLA).
The following research works provide a language or a framework that allows to formalize service level agreements (SLA).
However, they do not study how such SLAs can be used to monitor the service and evolve it as necessary.
WSLA~\cite{keller2003wsla} introduces a framework to define and break down customer agreements into a technical description of SLAs and terms to be monitored.
In ~\cite{mahbub2011translationsla}, a method is proposed to translate the specification of SLA into a technical domain directed in SLA@SOI EU project.
In the same project, \cite{comuzzi2009defavail} defines terms such as availability, accessibility and throughput as notions of SLA, however, the formal semantics and properties of the notions are not investigated.
In \cite{chen2007sladecompose}, authors describe how they introduce a function how to decompose SLA terms into measurable factors and how to profile them.
Timed automata is used in~\cite{raimondi2008fsmsla} to detect violations of SLA and formalize them.

Johnsen~\cite{johnsen2012modeling} introduce ``deployment components'' using Real-Time ABS~\cite{bjork2013:rtabs}.
A deployment component enables an application to acquire and release resources on-demand based on a QoS specification of the application.
A deployment component is a high level abstraction of a resource that promotes an application to a resource-aware level of programming.
Our work is distinguished by the fact that we separate the monitors from the application (service) themselves.
We argue that we aim to design the monitoring model to be as \emph{non-intrusive} as possible to the service runtime. 
Thus, we do not deploy the monitors inside the service runtime.

In Quanticol EU project\footnote{Quanticol EU project with no. 600708: \url{http://quanticol.eu/}}, authors in \cite{coles2011cost} and \cite{gilmore2011non} use statistical approaches to observe and guarantee service level agreements for public transportation.
We also present that service characteristics can be composed together. 
This means that evolving a system based on SLAs turns into a multi-object optimization problem.
In addition, in COMPASS EU project\footnote{COMPASS EU project with no. 287829: \url{http://www.compass-research.eu/}}, CML~\cite{woodcock2014contracts} defines a formal language to model systems of systems and the contracts between them.
CML studies certain properties of the model and their applications.
CML is used in the context of a Robotics technology to model and ensure how emergency sensors should react and behave according to the SLAs defined for them.
Our approach is similar to provide a generic model for service characteristics definition, however, we utilize timed and task automata.

