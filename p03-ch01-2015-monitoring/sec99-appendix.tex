
\appendix
\label{appendix}

\newpage

\section{Appendix: Previous write-ups}

\Behrooz{What follows may be obsolete. The content will be used/removed gradually. Do not remove.}

\subsection{Operational Semantics} % (fold)
\label{sub:operational_semantics}
We use operational semantics to present the semantics of the runtime monitoring. 
The global configuration is
$$
\mathcal{C} = \langle \mathcal{S}, \mathcal{A}, \mathcal{M}, P \rangle
$$
which respectively composes of the services, agreements, monitors, and platform.
Note that there is only one platform in this configuration.
For any actor in the system, we use a local configuration as $C = \langle Q, \sigma \rangle$ in which $Q$ is the queue of the messages of the actor and $\sigma$ is a local state that the actor may use for its responsibilities.

\paragraph{Instantiate a service} 
When a new service $s$ is instantiated, the global configuration makes a transition:
\[
\langle \mathcal{S}, \mathcal{A}, \mathcal{M}, P \rangle
\xrightarrow[m = \bfjtt{new} \jtt{ Monitor}(s, a)]{s=p \; ! \; \jtt{instantiate}(\jtt{name}, a)}
\langle \mathcal{S'}, \mathcal{A'}, \mathcal{M'}, P \rangle
 \\
\]
where $\mathcal{S'} = \mathcal{S} \cup \{s\}$, $\mathcal{A'} = \mathcal{A} \cup \{a\}$, and $\mathcal{M'} = \mathcal{M} \cup \{m\}$.
And, \jtt{name} is the name of the new service $s$ and $m$ are the monitor(s) established for $s$.

\paragraph{Monitor a service}
A monitor actors performs checks based on the agreements for a specific service.
The completion of a check on a service can yield an instance of measurement.
\[
\langle \mathcal{S}, \mathcal{A}, \mathcal{M}, P \rangle
\xrightarrow[p \; ! \; \jtt{publish}(\omega)]{\omega = m \; ! \; \jtt{check}()}
\langle \mathcal{S}, \mathcal{A}, \mathcal{M}, P' \rangle
\]
where monitoring platform $P$ state $\sigma$ is updated using the measurement: $\sigma_{P'} = \sigma_{P} \pm \omega$.

\paragraph{Allocate a new resource}
The platform instance $p$ is able to allocate a new resource $r$ to service $s$:
\[
\langle \mathcal{S}, \mathcal{A}, \mathcal{M}, P \rangle
\xrightarrow[]{r = p \; ! \; \jtt{allocate}(s) \; \bfjtt{duration}(t,t)}
\langle \mathcal{S}, \mathcal{A}, \mathcal{M'}, P' \rangle
\]
where $r$ is either the resource that is allocated to service $s$ or carries a failure for the allocation.
The allocation of a resource can be attributed with a deadline $t$ using ABS semantics where $t$ is the amount of time to expect $r$ is ready or failed.
If the operation succeeds, $\sigma_{P'} = \sigma_{P} \pm R(s) \cup \{r\}$ where $R(s)$ is the set of resources allocated to $s$.
% subsection operational_semantics (end)

% section modelling (end)

\subsection{A Timed Automata for Budget Compliance}
\label{sec:prop:budget}

Similar to Section \ref{sec:service:avail:fsm}, we use a timed automata (see Definition \ref{def:timed:automata}) to model the behavior of service budget compliance $\beta(s,\tau)$ (see Definition \ref{def:budget:compliance}) represented in Figure \ref{fig:fsm:budgetcomp}.
% 
\begin{figure}[h]
\centering
\begin{tikzpicture}[->,>=stealth',shorten >=1pt,auto,node distance=4cm,thick]
\node[state,accepting]        (A)   {$q_B$};
\node[state]  [right of=A]    (N)   {$q_{\overline{B}}$};
\path
(A) edge    [loop left, left]  node    {$\delta_M^1=(\bot,g_B,\tau)$}     (A)
(A) edge    [bend right, below]       node    {$\delta_M^2=(\dot{a},g_{\overline{B}},\tau)$}     (N)
(N) edge    [loop right, right]  node    {$\delta_M^3=(\dot{a},g_{\overline{B}},\tau)$}     (N)
(N) edge    [bend right, above]       node    {$\delta_M^4=(\bot,g_B,\tau)$}     (A)
;
\end{tikzpicture}
\caption{A timed automata for Budget Compliance $\beta(s,\tau)$}
\label{fig:fsm:budgetcomp}
\end{figure}
% 

$M_\beta$ has two states, $Q = \{q_B, q_{\overline{B}} \}$.
In comparison, $M_\beta$ is simpler than $M_\alpha$.
Computation of the cost of a service is independent of whether the service is faulty or not responsive.
Thus, $M_\beta$ does not have a state of $q_e$.
The invariants that guard transitions in $M_\beta$ are as follows:
\begin{align}
g_B \; &: \; 0 \geq \beta(s,\tau) < \varepsilon_{\euro} \\
g_{\overline{B}} \; &: \; |\beta(s,\tau)| > \varepsilon_{\euro}
\end{align}
The transitions of $M_\beta$ can be described similar to that of $M_\alpha$.
The different is in the action $\dot{a}$ produced.
An action in $M_\beta$ is a message to monitoring platform $p$ that deallocates a resource from services $s$:
\[
p \; ! \; \jtt{deallocate(s)}
\]
that terminates one resource allocated to $s$ from platform $p$.
This action is immediate and never fails.

\section{Distributed Service Sustainability} % (fold)
\label{sec:distributed_service_sustainability}
\begin{defn}[Service Sustainability $\gamma(s,\tau)$]
\label{def:service:sustain}
We define the service sustainability as:
\[
\gamma(s,\tau) = \alpha(s,\tau,t_c) \times \beta(s,\tau)
\]
that is the composition of service availability $\alpha(s,\tau,t_c)$ and budget compliance $\beta(s,\tau)$ over the time span $\tau$.
Note that under a fixed agreement $a$ for $s$, $\alpha(s,t)$ and $\beta(s,t)$ have an inverse relation to one another:
\[
\alpha(s,t) \uparrow \iff \beta(s,t) \downarrow
\]
because consuming resources for service $s$ for $\alpha$ mean more cost for $s$ under the same agreement $a$ that leads to lower $\beta$ (and vice versa).
\end{defn}

% section distributed_service_sustainability (end)

\begin{theorem}[Budget Compliance Convergence]
\label{thm:budget:comp:conv}
The automata $M_\beta$ on budget compliance $\beta(s,\tau)$ makes a transition to state $q_B$ in a finite amount of time under Assumptions \ref{def:single:resource:type} and \ref{def:resource:cost:constant} given that $\tau \geq \hat{t}$ and agreement $a$ for service $s$ is fixed and not changing through time.
\end{theorem}

\begin{proof}[Budget Compliance Convergence]
Proof can be completed using same technique in Theorem \ref{thm:service:avail:conv}.
\newline$\Box$
\end{proof}

\subsection{A Timed Automata for Service Sustainability}
\label{sec:sustain}

The timed automata $M_\gamma$ for $\gamma(s,\tau)$ is constructed by $M_\alpha \times M_\beta$.
$M_\gamma$ has two states, $Q = \{ q_S, q_{\overline{S}}\}$.
$M_\gamma$ enters its final state $q_S$ when $M_\alpha$ is in $q_A$ and $M_\beta$ is in $q_B$.
$M_\gamma$ enters its non-final state $q_{\overline{S}}$ when either $M_\alpha$ is not in $q_A$ or $M_\beta$ is not in $q_B$.

\begin{theorem}[Sustainable Service Convergence]
\label{thm:sustain:conv}
The sustainability of service $s$ converges to:
$$
\lim_{T \to \infty} \; \gamma(s, \tau) = (E(s,\tau) - \varepsilon) \times (E_B(s,\tau) - \varepsilon_{\euro})
$$
if and only if, agreement $a$ for service $s$ is \emph{modifiable} from a monitoring window $\tau$ to the next given that either $\alpha(s,\tau,t_c)$ or $\beta(s,\tau)$ fails in one $\tau$.
\end{theorem}

\begin{proof}[Sustainable Service Convergence]
In other words, $M_\gamma$ remains in state $q_S$ as long as both $\alpha(s,\tau,t_c)$ and $\beta(s,\tau)$ are compliant.
$M_\gamma$ does not remain in state $q_S$ if $\alpha(s,\tau,t_c)$ or $\beta(s,\tau)$ change under a fixed agreement $a$.
The proof is by contradiction.
We study different cases each of which has a \emph{fixed} element and we conclude a contradiction. 

Let $M_\alpha$ be in $q_A$ and $M_\beta$ in $q_B$ at a given time $T$.
At time $T$, if $g_{\overline{A}}$ holds, $M_\alpha$ makes a transition to $q_{\overline{A}}$ and starts to allocate a new resource $r$ for $s$.
At a later time $T' \geq T + \tau$, $M_\alpha$ makes a transition to $q_A$ based Theorem \ref{thm:service:avail:conv}.
However, if $g_{\overline{B}}$ holds at time $T'$, $M_\beta$ makes a transition to $q_{\overline{B}}$ because the newly allocated resource $r$ makes budget compliance $\beta(s,\tau)$ violate.
Based on Theorem \ref{thm:budget:comp:conv}, $M_\beta$ makes a transition to $q_B$ after deallocating at least one resource from $s$ at a later time $T'' \geq T' + \tau$.
At time $T''$, the system recurs the behavior at time $T$ and thus $\gamma(s,\tau)$ for every subsequent $\tau$ cannot converge to $(1-\varepsilon)\times(1-\varepsilon_{\euro})$.
Similarly, it can be reasoned how changes in $M_\beta$ lead to changes in $M_\alpha$ and eventually destabilize $M_\gamma$.

The proof is complete if the modification to be applied to $a$ of $s$ is calculated at a given monitoring window $\tau$.
When $M_\alpha$ makes a transition to $q_{\overline{A}}$, the action is to allocate a new resource $r$ to $s$.
We complement the action by another one.
At time $T$, if pre-calculation of $\euro(s,\tau)$ reveals violation of $\beta(s,\tau)$, we update $B(s,\tau)$ by $B(s,\tau) = B(s,\tau) + \overline{\euro}_r \times \tau$; i.e. the cost difference of a single resource over monitoring window is reflected into service budget.
At time $T'$, $M_\beta$ does not violate because of the newly added resource because the service budget already includes the necessary changes to pass.
Similarly, when $M_\beta$ makes a transition to $q_{\overline{B}}$, the action is to deallocate a resource from $s$.
We complement the action by another one.
At time $T$, if pre-calculation of $\alpha(s,\tau,t_c)$ reveals a violation, we update $E(s,\tau)$ by $E(s,\tau) = E(s,\tau) + \Delta(s,\tau)$; i.e. the availability distance caused by deallocation single resource over monitoring window is reflected into service expectation.
At time $T'$, $M_\alpha$ does not violate because of the adjustment. 
Thus, in either case, when $M_\alpha$ or $M_\beta$ make a transition to an unacceptable state, we are able to make an adjustment to the agreement $a$ of service $s$ such that the convergence of $\gamma(s,\tau)$ during the next monitoring window is preserved. 
\newline
$\Box$
\end{proof}

\section{Application of Actor-based Monitoring Model}
\label{sec:app}

If we expand $\gamma(s,\tau)$, we have:
\begin{align*}
\gamma(s,\tau) &= \alpha(s,\tau,t_c) \times \beta(s,\tau) \\
 &= \frac{|\, \{\; \omega \; | \; 0 \leq v(\omega) - e(s(\omega),a(\omega)) < \epsilon \; \}\,|}{|\{\omega\}|}
 \times \frac{\euro(s,\tau) - B(s,\tau)}{B(s,\tau)} \\
 &=_{T \rightarrow \infty} (E(s,\tau) - \varepsilon) \times (E_B(s,\tau) - \varepsilon_{\euro})
\end{align*}
in which the parameters that can possibly change for service $s$ are:
\begin{description}
\item[Agreement Expectation $e(s, a)$] defines the values that are expected for each agreement $a$ of service $s$.
\item[Service Budget $B(s,\tau)$] defines the limit of the budget that can be spent for service $s$ in a period of time $\tau$.
\item[Service Availability Tolerance $\varepsilon$] defines the accepted deviation of service availability calculated for $s$ over a period of $\tau$.
\item[Budget Compliance Tolerance $\varepsilon_{\euro}$] defines the accepted deviation of budget compliance calculated for $s$ over a period of $\tau$.
\item[Monitoring Window $\tau$] define the time span $\tau$ with which all monitoring checks are performed and based on which measurements, the above properties are calculated and reacted upon.
\end{description}
In the following, we discuss how changes of the above parameters affect service sustainability and how they can be used to synthesize or drive the generation of monitoring model for a service based on its agreements.

\subsection{Tolerance Flexibility $\varepsilon$ and $\varepsilon_{\euro}$}

Theorem \ref{thm:sustain:conv} proved that $\gamma(s,\tau)$ converges to $(E(s,\tau)-\varepsilon)\times(E_B(s,\tau)-\varepsilon_{\euro})$ given certain conditions.
The values of $E(s,\tau)$ and $E_B(s,\tau)$ are bound to $1$.
To simplify, we define $f(\varepsilon,\varepsilon_{\euro}) = (1 - \varepsilon)\times(1-\varepsilon_{\euro})$.
On its own, this function is interesting.
Figure \ref{fig:tolerance:func} presents the landscape of this function with $\varepsilon$ and $\varepsilon_{\euro}$ as its variables.
% 

The following summarizes the characteristics of the function and how it affects $\gamma(s,\tau)$:
\begin{itemize}
\item The following property holds:
\[
(1-\varepsilon) \times (1-\varepsilon_{\euro}) = 1 \iff \varepsilon = 0 \;\lor\; \varepsilon_{\euro} = 0
\]
This property shows that choosing $0$ value for $\varepsilon$ or $\varepsilon_{\euro}$ makes $\gamma(s,\tau)$ more strict and less tolerant. 
Unless it is not a hard business requirement, the tolerance parameters must allow for room of error for either $\alpha(s,\tau,t_c)$ or $\beta(s,\tau)$; otherwise, the sustainability $\gamma(s,\tau)$ exposes more fragility.
\item Similarly, choosing $\varepsilon = 1$ or $\varepsilon_{\euro} = 1$ means that respectively $\alpha(s,\tau,t_c)$ and $\beta(s,\tau)$ is \emph{not} a business driver.
In this setting, the service sustainability $\gamma(s,\tau)$ depends on a single property (either $\alpha(s,\tau,t_c)$ or $\beta(s,\tau)$) which makes its maintenance easier.
Additionally, choosing such model allows to design a business in which exclusively either service availability or budget compliance is crucial but not both.
\item Apart from the corner values, the value of the function is \emph{confined} by the arcs that are projected on the ceiling of the box in Figure \ref{fig:tolerance:func}.
This means that the function values changes co-dependently based on the changes of $\varepsilon$ and $\varepsilon_{\euro}$; i.e. optimization of $\gamma(s,\tau)$ is freely achievable by changing each variable.
\item It should be also noted that the values for $\varepsilon$ and $\varepsilon_{\euro}$ should \emph{not} be so relaxed; 
i.e. values higher than $0.1$ would create a large range of error which may not be desired.
\end{itemize}


\subsection{Business Constraints $E_B(s,\tau)$ and $E(s,\tau)$ for Monitor Generation}

The parameters $E(s,\tau)$ and $E_B(s,\tau)$ are significantly driven from a business perspective.
Such parameters enforce business-level constraints on the services that are also bound from a legal perspective.
In the past decade, there is a considerable research on how to specify and model such business constraints generally referred to as service level agreements (SLA).
A number of languages, frameworks and models have been developed to be able to model and specify SLAs for the deployment of services.
Among others, we refer to SLA$\star$~\cite{kearney2010sla}, WSLA~\cite{keller2003wsla}, and SLAng~\cite{lamanna2003slang}.
The focus of this research is not enter this domain and yet build on top it a model to be able to reason about ``distributed service properties'' such as service availability or budget compliance. 
In the following, we present a set of pseudo algorithms for the model in Section \ref{sec:modelling} that present how the monitoring can be generated based on the agreements that are specified. 
In Algorithm \ref{alg:gen:monitor}, the monitoring actor method \jtt{check} is presented that collects the measurements and publishes them to the platform.



% 
% section service_synthesis_based_monitoring_model_parameters (end)

\subsection*{A nice picture}

The top surface (including the arcs) is the 2-D projected values for $\Gamma$.
% 
\begin{figure}[h]
\centering
\begin{tikzpicture}
\begin{axis}[
  view={45}{45},
  colorbar,
  3d box=complete,
  enlargelimits=false,
  domain=0:1,
  xlabel=$\varepsilon_\alpha$,
  ylabel=$\varepsilon_\beta$,
  zlabel=$\Gamma$
]
\addplot3[surf]
  {(1-x) * (1-y)};
\addplot3[contour gnuplot={contour dir=z, labels=false, draw color=black}, thick, z filter/.code=\def\pgfmathresult{1}]
  {(1-x) * (1-y)};
\end{axis}
\end{tikzpicture}
\caption{
The landscape for $\Gamma$.
} 
\label{fig:sustain:factor}
\end{figure}
% 

\subsection{Generalization of $\alpha$, $\beta$ and $\gamma$}

For every characteristics of the service, define a predicate function:

\[
F_C(s,\tau, C) \in \{0, 1\}
\]

in which $0$ is \jtt{false}, and $1$ is \jtt{true}.
$C$ denotes the set of properties that is defined for the service $s$ for the predicate function.

For example, in the context of $\alpha$, the predicte function is:

\[
F^\alpha_{C_\alpha}(s, \tau, \{E(s,\tau), \varepsilon_\alpha\}) = \begin{cases}
0  &\;  \alpha(s,\tau,t_c) < 1 - \varepsilon_\alpha \\
1 &\; \textnormal{otherwise}
\end{cases}
\]

We gather the set of all characteristics for a service $s$ by
\[
\mathcal{C} = \{C_1, \ldots, C_n\}
 \]
then for each $C_i$, we define the predicate function $F_{C_i}$.
Similar to $M_\alpha$ and $M_\beta$, we build a timed automata for each $C_i$ using $F_{C_i}$. 
Thus, we build a set of timed automata
\[
\mathcal{M} = \{M_{c_i}, \ldots, M_{c_n}, M_P\}
\] 

$\mathcal{M}$ can be partitioned in two \emph{disjoint} subsets, $\mathcal{M}^+$ and $\mathcal{M}^-$, such that $\mathcal{M}^+$ only containts those that use platform $P$ API to \jtt{allocate} a new resource to service $s$ whereas $\mathcal{M}^-$ only contains those that use platform $P$ API to \jtt{deallocate} a resource from service $s$:
\begin{itemize}
  \item $\forall \mathcal{M}_k \in \mathcal{M}^+$ Thoerem \ref{thm:service:avail:conv} holds for $\mathcal{M}_k$.
  \item $\forall \mathcal{M}_k \in \mathcal{M}^-$ Theorem \ref{thm:service:budget:conv} holds for $\mathcal{M}_k$.
  \item $\forall \mathcal{M}_k \in \mathcal{M}^+ \; , \mathcal{M}_l \in \mathcal{M}^- \;, \mathcal{M}_m = \mathcal{M}_k \times \mathcal{M}_l$ Theorem \ref{thm:service:sust} holds for $\mathcal{M}_m$. 
\end{itemize}

\subsection{A Timed Automata for Platform} 
\label{sec:automata:platform}

To model the behavior of the platform $P$ with a timed automata, we use the notion of ``deadline'' from Real-Time ABS~\cite{JaghooriBCS09,bjork2013:rtabs}.
A message is expected to complete (either successfully or with failure) within a specific amount of time, the deadline.
We choose monitoring window $\tau$ as the deadline for any message that is internally processed by platform $P$.
In practice, a deadline for a message is part of the implementation details of any platform API for an IaaS provider.
Figure \ref{fig:fsm:platform} presents a timed automata for platform $P$.

In the timed automata that we design for different elements of the system, the following apply.
Every timed automaton has only one state which is the start and end state.
We use $\dot{c}$ to denote the clock of the timed automaton.
The timed automaton makes a transition at a random time in every monitoring window $\tau$.
A transition may have two forms: \emph{periodic} by the clock or \emph{reactive} through an external event or API call.
For example, $M_P$ has a trasition for \jtt{allocate} a new resource which is reactive to an external platform API call.
And, it has a transition for updating service state $\sigma(s)$ which is periodic and regulated by the clock. 
Every reactive transition has a fixed deadline equal to $\tau$.
A reactive transition is presented with a ``dotted'' line in Figure~\ref{fig:fsm} whereas a periodic transition is presented with a ``solid'' line.
Every transition is presented as $\frac{\textsf{condition}}{\textsf{action}}$.
The \textsf{condition} is a boolean expression which evaluates to either \jtt{true} or \jtt{false}.
The \textsf{action} is an atomic operation or an asynchronous message using Real-Time ABS with a fixed deadline equal to $\tau$.
For every timed automaton, there is an implicit transition as: $\frac{\dot{c} \geq \tau}{\dot{c} \gets 0}$.

Every transition in $M_P$ has a deadline of $\tau$;
it is expected that the result of the action of the transition is completed before the end of the next monitoring window.
In particular, this implies that the state of a service ($\sigma(s)$), and the history of a service ($h(s)$) are updated every monitoring window.

Capturing a failure in the platform is achieved by identifying a resource $r$ for some service $s$ such that the resource is recorded in the history of the service, but is not any more in the current state $\sigma$ of the service.

The platform also receives requests to allocate and deallocate specified by its API. 
All such requests use the same deadline $\tau$.

% 
\begin{figure}[t]
\centering

% M_P
\subcaptionbox{{\small{$M_P$: Timed Automaton for Platform $P$}}\label{fig:fsm:platform}}
{
\begin{tikzpicture}[->,>=stealth',auto,node distance=6cm,thick,scale=0.7,every node/.style={scale=0.7}]
\node[state,accepting]   
    (s)   {{\small$\dot{c} < \tau$}};
% \node[state]   [right of=s]      (e)   {$\dot{c} = t$};
\path[every loop/.style={looseness=7}]
(s) 
    edge    
    [loop left,style=densely dotted]  
    node {$\jtt{allocate}(s,n) \; \bfjtt{deadline}(\tau)$}          
    (s)
(s) 
    edge    
    [loop right,style=densely dotted]   
    node {$\jtt{deallocate}(s) \; \bfjtt{deadline}(\tau)$} 
    (s)
(s)
    edge
    [loop,out=150,in=120,looseness=10]
    node {$\sigma(s) \gets \jtt{getState}(s) \; \bfjtt{deadline}(\tau)$}
    (s)
(s)
    edge
    [loop,in=30,out=60,looseness=10]
    node  {$h(s) \gets \jtt{getHistory}(s) \; \bfjtt{deadline}(\tau)$}
    (s)
(s)
    edge
    [loop above]
    node {$\frac{\forall s\; \exists r: \; r \in h(s) \wedge r \not\in \sigma(s)}{\bfjtt{skip}}$}
    (s)
;
\end{tikzpicture}
}

% M_alpha
\subcaptionbox{{\small{$M_\alpha$: Timed Automaton $\alpha(s,\tau,t_c)$}}\label{fig:fsm:serviceavail}}
{
\begin{tikzpicture}[->,>=stealth',auto,node distance=6cm,thick,scale=0.7,every node/.style={scale=0.7}]
\node[state,accepting]
(s)   
    {{\small$\dot{c} < \tau$}};
\path[every loop/.style={looseness=7}]
(s) 
    edge    
    [loop left]  
    node {$\frac{\alpha_\sigma(s,\tau) < 1 - \varepsilon_\alpha}{P \; ! \; \jtt{allocate}(s,n_r)}$}          
    (s)
(s) 
    edge    
    [loop right]   
    node {$\frac{1 - \varepsilon_\alpha \leq \alpha_\sigma(s,\tau) \leq 1 + \varepsilon_\alpha}{h(s) \gets \sigma(s)}$} 
    (s)
;
\end{tikzpicture}
}

% M_beta
\subcaptionbox{{\small{$M_\beta$: Timed Automaton for $\beta(s,\tau)$}}\label{fig:fsm:budgetcompliance}}
{
\begin{tikzpicture}[->,>=stealth',auto,node distance=6cm,thick,scale=0.7,every node/.style={scale=0.7}]
\node[state,accepting]   (s)   {$\dot{c} < \tau$};
\path[every loop/.style={looseness=10}]
(s) 
    edge    
    [loop left]  
    node {$\frac{\beta(s,\tau) < 1 - \varepsilon_\beta}{P \; ! \; \jtt{deallocate}(s)}$}          
    (s)
(s) 
    edge    
    [loop right]   
    node {$\frac{1 - \varepsilon_\beta \leq \beta(s,\tau) \leq 1 + \varepsilon_\beta}{\sigma(s) \gets h(s)}$} 
    (s)
;
\end{tikzpicture}
}

% M_gamma
\subcaptionbox{{\small{$M_\gamma$: Timed Automaton for $\gamma(s,\tau)$}}\label{fig:fsm:sustain}}
{
\begin{tikzpicture}[->,>=stealth',auto,node distance=6cm,thick,scale=0.7,every node/.style={scale=0.7}]
\node[state,accepting]   (s)   {$\dot{c} < \tau$};
\path[every loop/.style={looseness=10}]
(s) 
    edge    
    [loop left]  
    node {$\frac{\alpha_\sigma(s,\tau) < 1 - \varepsilon_\alpha}{P \; ! \; \jtt{allocate}(s,n_r)}$}          
    (s)
(s) 
    edge    
    [loop right]   
    node {$\frac{1 - \varepsilon_\alpha \leq \alpha_\sigma(s,\tau) \leq 1 + \varepsilon_\alpha}{h(s) \gets \sigma(s)}$} 
    (s)
(s) 
    edge    
    [loop,in=30,out=60,looseness=10]  
    node {$\frac{\beta(s,\tau) < 1 - \varepsilon_\beta}{P \; ! \; \jtt{deallocate}(s)}$}          
    (s)
(s) 
    edge    
    [loop,in=120,out=150,looseness=10]   
    node {$\frac{1 - \varepsilon_\beta \leq \beta(s,\tau) \leq 1 + \varepsilon_\beta}{\sigma(s) \gets h(s)}$} 
    (s)
;
\end{tikzpicture}
}


\caption{Timed automata for elements of the monitoring platform}
\label{fig:fsm}
\end{figure}

\subsection{A Timed Automata for Service Availability}
\label{sec:service:avail:fsm}

We use timed automata~\cite{alur:1994:timedautomata} $M_\alpha$ to manage service availability (see Definition \ref{def:service:availability}).
$M_\alpha$ is designed to specifically address an \emph{under-capacity} service:

The timed automata $M_\alpha$ has two \emph{possible} transitions.
%A transition is chosen according to the condition check. 
If the condition evaluates to \jtt{true}, the corresponding action is taken and the clock is reset.
An action in $M_\alpha$ is an \emph{asynchronous message} in the monitoring model in ABS or an update to $\sigma(s)$ or $h(s)$.
Below we explain the possible conditions and associated actions:

\textbf{Available Service}
  When service $s$ is available, the action is to update the history of the service to record the number of resources allocated to $s$. 
  Specifically, $h(s)$ is the last observed number of resources that ensured $s$ was available.
  This is captured in the right transition in $M_\alpha$.

\textbf{Under-capacity Service}
  When service $s$ is not available, the action is to allocate \emph{at least} one resource.
  This strategy can be improved since we captured the number of resources that
  most recently ensured availability ($h(s)$). Hence, the number of resources to allocate to service $s$, can be calculated as:
  \[
  n_r = 
  \begin{cases}
    \mathsf{max}(1, h(s)-\sigma(s)) &\; h(s) > 1 \\
    1 &\; \text{otherwise}
  \end{cases}
  \]
  Intuitively, when there is no history for $s$, \emph{one} resource is allocated.
  As soon as $h(s)$ is recorded, the required number of resources is the difference between the current number of resources and the last number that ensured $s$ is available.
  This is captured in the left transition in $M_\alpha$.

The automaton $M_\alpha$ guarantees the following important property:
% present an interesting property of $M_\alpha$ in the following theorem.
% 

\begin{defn}[Service Availability Convergence]
\label{thm:service:avail:conv}
Given that: 
\begin{itemize}
  \item platform $P$ provides resources with no limits (Assumption~\ref{def:platform:cap}), and,
  \item monitoring window $\tau$ is chosen such that it's larger that any resource initialization time ($\tau > \hat{t}$) (Assumption~\ref{def:resource:init:time}), 
  \item the number of failed resources
  %in every monitoring window $\tau$
  is bounded by a constant, 
\end{itemize}
then, the timed automata $M_\alpha$ for service availability $\alpha(s,\tau,t_c)$ ensures that service $s$ is available in a finite amount of time.
\end{defn}

\begin{proof}%[Service Availability Convergence]
By induction on $n_r$.
% $\Box$
\end{proof} % end thm:service:avail:conv

\subsection{A Timed Automata for Service Budget Compliance}
\label{sec:automata:budget}

Analogously, we define a timed automata $M_\beta$ for service budget compliance (Definition~\ref{def:budget:compliance}). The automaton is shown in Figure~\ref{fig:fsm:budgetcompliance}.  $M_\beta$ is specifically designed to address a service that is \emph{over-budget}.
% 

$M_\beta$ has three possible condition with corresponding actions:

\textbf{Budget Compliant Service}
  When service $s$ is budget-compliant, the action is to update the history of the service to record the current number of resources allocated to $s$. 
  Formally, $h(s)$ denotes the last observed number of resources that ensured $s$ was budget-compliant.

\textbf{Over-budget Service}
  When service $s$ is over-budget and not compliant, the action is to deallocate \emph{one} resource.
  We do not maximize deallocation to prevent sudden service unavailability.
  % \item[Failure in Platform]
  % We capture failures in the platform $P$ by deallocating a resource that is known to be allocated to service $s$ but is not recognized as operational by the platform itself.
  % There is no reaction to platform failures for budget compliance, because when a resource is deallocated from a service, it increases the budget compliance per cut-down on cost.
  % Platform failures \emph{can} lead to a reaction in the previously discussed automaton $M_\alpha$ to ensure service availability.

The above definition of $M_\beta$ guarantees the following important property:
%exposes an interesting property that is presented in the following theorem.

\begin{defn}[Service Budget Compliance Convergence]
\label{thm:service:budget:conv}
The timed automata $M_\beta$ for service budget compliance $\beta(s,\tau)$ ensures that service $s$ is budget compliant in a finite amount of time.
\end{defn}

\begin{proof}%[Service Budget Compliance Convergence]
Similar to that of Theorem~\ref{thm:service:avail:conv}.
\end{proof}

\subsection{Service Sustainability}
\label{sec:service:sustain}

In the previous two sections, we modelled service availability $\alpha(s,\tau,t_c)$ and service budget compliance $\beta(s,\tau)$ using two timed automata.
In isolation, the automata guarantee two important theorems: Theorem \ref{thm:service:avail:conv} and \ref{thm:service:budget:conv}.
%  The automata two theorems on the two behave in the absence of the other limitation.
%In reality,
In practice, the main challenge is to be able to ensure both $\alpha(s,\tau,t_c)$ and $\beta(s,\tau)$ at the same time.
However, recall that these two properties of a service $s$ have conflicting goals: $\beta(s,\tau)$ decreases as $\alpha(s,\tau,t_c)$ increases (and vice versa).
%as an \emph{inverse} relation to one another.
