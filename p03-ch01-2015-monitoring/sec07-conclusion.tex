% -*- root: paper-esocc.tex -*-

\section{Future work}
\label{sec:conclusion}


%In this paper, we introduced a distributed monitoring model in order to evolve a service towards its SLA. 
%We formalized characteristics of a distributed service: service availability $\alpha(s,\tau,t_c)$ and service budget compliance $\beta(s,\tau)$.
%We designed task and timed automata for each characteristic to react to the changes of the service in the environment to re-establish its SLA.
%% We presented theorems how SLA for service availability and budget compliance can be met.
%We composed the two characteristics as service sustainability.
%% We showed how the schedulability analsysis of service sustainability builds a model.
%The monitoring model can be used to analyze, model check, and optimize the input parameters to the system (such as $\tau$ and $t_G$), leveraging well-established model checking and schedulability analysis techniques.
%We evaluated the monitoring model on the SDL Fredhopper cloud services.
%We discussed how the results of the evaluation confirm the theorems presented in Section~\ref{sec:timed:fsm} from a practical standpoint.

% Using the formal model of the service characteristics, we design a monitoring model using actors. 
% Each monitor in the system is responsible for one service.
% Monitors use asynchronous communication through Real-Time ABS to interact with the platform API.
% The monitors observe the system and react to the changes or expectations of the properties. 
% They make changes to the system (\jtt{allocate} or \jtt{deallocate}) through Platform API to ensure the system satisfies its SLA.
% We presented an evaluation of the implementation of the  monitoring platform.
% We showed how choosing certain parameters in the system influences the evolution of the services at runtime.

% There is a wealth of future work for this research.
We continue to generalize the notion of the distributed service characteristics and investigate how the composition of an arbitrary number of such properties can be formalized and reasoned about.
In the context of the ENVISAGE project, industry partners define their service characteristics in this framework and monitor the service evolution.
Moreover, the work will be extended to generate parts of the monitoring platform based on an input of different SLA formalizations such as SLA$\star$~\cite{kearney2010sla}.
Currently, we are integrating our automated monitoring infrastructure
into the in-production SDL Fredhopper cloud services (cf. Section~\ref{sec:fredhopper_example}).
