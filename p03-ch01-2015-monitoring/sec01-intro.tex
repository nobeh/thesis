% -*- root: paper-podc.tex -*-

\section{Introduction}
\label{sec:introduction}

Cloud computing provides the elastic technologies for virtualization. 
Through virtualization, software itself can be offered as a service (Software as a Service,
SaaS). 
One of the aims of SaaS is to allow service providers to offer reliable software services while scaling up and down allocated resources based on their availability, budget, service throughput and the Service Level Agreements (SLA).
Thus, it becomes essential that virtualization technologies facilitate elasticity in a way that enables business owners to \emph{rapidly} evolve their systems to meet their customer requirements and expectations.

The fundamental technical challenge to a SaaS offering is maintaining the quality
of service (QoS) promised by its SLA. In SaaS, providers must
ensure a consistent QoS in a dynamic virtualized environment with variable
usage patterns. Specifically, virtualized environments such as the cloud
provide elasticity in resource allocation, but they often do not offer an SLA that can
guarantee constant resource availability. As a result, SaaS providers are
required to react to resource availability at runtime. Furthermore, by offering a
$24/7$ software service, SaaS providers must be able to react to certain service
usage patterns, such as an increase in throughput to ensure the SLA is
maintained.

Runtime monitoring~\cite{Logean_monitoring,BratanisDS10} is a dynamic analysis approach
based on
%observing the runtime of a software system and
extracting relevant information about the execution.
Runtime monitoring may be employed to collect statistics about the service usage over time, and to detect and react to service behavior. 
This latter ability is fundamental in the SaaS approach to guarantee the SLA of a service
% to the provision of virtualized services
%This ability
and is the focus of this paper.

The monitoring model that is presented in this paper is designed to \emph{observe} 
% the evolution 
in real-time certain  service characteristics and \emph{react} to them to ensure the evolution of the system towards its SLA.
% In this paper we develop an actor-based monitoring model to study the dynamic
% provisioning of resources to services for maintaining service availability given a
% certain level of budget compliance. Our monitoring model formalizes the notion of service
% sustainability, that is, whether a reliable service can be maintained that
% meets the SLA.
%Specifically, our actor-based model is defined in terms of the Abstract
%Behavioral Specification (ABS) language~\cite{johnsen2012abs} and Real-Time ABS~\cite{johnsen2012modeling}, %that is
%an extension of ABS that allows modeling resources and measuring resource usage.
%%providing scheduling of time concerns for ABS. 
%ABS is an object oriented modeling
%language with a formal executable semantics that is designed for modeling
%highly adaptable distributed concurrent systems. 
%%We provide an operational semantics to our model, and 
%% We formalize the notion of service availability, budget compliance, and service sustainability.
%%We apply the formalization results to present
%% 
%We choose actor model~\cite{agha86book} to base the monitoring model on.
%The monitoring model is required to be reactive to changes; 
%i.e. synchronization risks must be avoided as much as possible.
Asynchronous communication is an essential feature of a monitoring model 
in a distributed context.
Asynchronous communication accomplishes non-intrusive observations of the service runtime. 
Further, the monitoring model is expected to operate according to certain real-time constraints specified by the SLA of the service.
Satisfying the real-time constraints is the main challenge in a distributed monitoring model.

%  based on a  scheduling policy;
% i.e. it should be able to use application-level scheduling mechanisms~\cite{nobakht2013future} to meet its requirements.
% A basic assumption underlying our monitoring model is that gathering service statistics and information at runtime should not affect the runtime behavior of the services themselves.

%In addition, the monitoring model should operate in a distributed environment of services.
%Distributed remote monitors operate
%%against their configured
%services to
%%guard them
%ensure their SLAs.

In this paper, we formalize service availability and budget compliance in a distributed deployment environment.
This formalization is based on  
high-level task automata models~\cite{alur:1994:timedautomata,fersman2007task,jaghoori2010time}.
The automata capture the real-time evolution of the resources provided by a distributed deployment platform and 
the above two main service characteristics.
These  task  automata represent the  real-time generation of the asynchronous events 
extended with deadlines~\cite{bjork2013:rtabs,nobakht2013future}
by the monitoring platform for managing resources (i.e. allocation or  deallocation).
The main result of this paper is a formal model to optimize and reason about the above service  characteristics through monitoring.
In particular, the \emph{schedulability} of the underlying timed automata implies service availability and budget compliance.
%present how the semantics of the timed automaton evolves the system towards its service level agreements.
%We also present the semantics of the distributed environment using a separate timed automaton.
%We prove theorems on the above timed automata and how their semantics guarantee achievement of the service level agreements for service availability and service budget compliance.
%In the semantics of the timed automata, we utilize asynchronous communication along with deadlines for messages.
Furthermore, we introduce a  composition of  service availability and  budget compliance which captures service sustainability.
We show that service sustainability presents a multi-objective optimization problem.
% and to be able to ensure this property, elements of the monitoring platform needs to changes during the runtime of the system.
% We show how they can be used in the application in business levels and how to design and generate corresponding monitors.

%The monitoring model that is presented in this research is able to \emph{observe} a system based on the above service characteristics and \emph{react} to them to ensure the evolution of the system towards its SLA.
%% In this paper we develop an actor-based monitoring model to study the dynamic
%% provisioning of resources to services for maintaining service availability given a
%% certain level of budget compliance. Our monitoring model formalizes the notion of service
%% sustainability, that is, whether a reliable service can be maintained that
%% meets the SLA.
%Specifically, our actor-based model is defined in terms of the Abstract
%Behavioral Specification (ABS) language~\cite{johnsen2012abs} and Real-Time ABS~\cite{johnsen2012modeling}, %that is
%an extension of ABS that allows modeling resources and measuring resource usage.
%%providing scheduling of time concerns for ABS. 
%ABS is an object oriented modeling
%language with a formal executable semantics that is designed for modeling
%highly adaptable distributed concurrent systems. 
%%We provide an operational semantics to our model, and 
%% We formalize the notion of service availability, budget compliance, and service sustainability.
%%We apply the formalization results to present
%% 
%We choose actor model~\cite{agha86book} to base the monitoring model on.
%The monitoring model is required to be reactive to changes; 
%i.e. synchronization risks must be avoided as much as possible.
%Asynchronous communication allows the monitoring model to be concurrent and reactive.
%The monitoring model is expected to expose certain properties based on a time schedule;
%i.e. it should be able to use application-level scheduling mechanisms~\cite{nobakht2013future} to meet its requirements.
%The monitoring model should be non-intrusive and distributed. 
%Gathering service statistics and information at runtime should not affect the runtime behavior of the services themselves.
%In addition, the monitoring model should operate in a distributed environment of services.
%Distributed remote monitors operate
%%against their configured
%services to
%%guard them
%ensure their SLAs.

%The rest of the paper is structured as follows: 
%In Section \ref{sec:relatedwork} we discuss related work in relation with runtime monitoring and service level agreements.
%We provide a real-life example from SDL Fredhopper in Section \ref{sec:fredhopper_example} that introduces an industrial case from which the model is extracted and on which it is evaluated. 
%We present the elements of a distributed environment in Section \ref{sec:modelling} and then formalize necessary assumptions and definitions.
%In Section~\ref{sec:timed:fsm}, we present the timed automata for service characteristics and theorems.
%In Section~\ref{sec:implementation}, we discuss the evaluation of the monitoring model.
%Section~\ref{sec:conclusion} presents the future work and concludes the paper.
