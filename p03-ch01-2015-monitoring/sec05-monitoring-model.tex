% -*- root: paper-esocc.tex -*-

\section{Service Characteristics Verification}
\label{sec:timed:fsm}

In this section, we use timed automata and task automata to model the behavior of a monitoring platform $P$, the deployment environment $E$, and the monitoring components for service availability $\alpha(s,\tau,t_c)$ and budget compliance $\beta(s,\tau)$. 
\cite{jaghoori2010time} defines a task automata as an extension of timed automata in which each task is a piece of executable program with $(b,w,d)$: best/worst time and deadline of the task. A task automata uses a scheduler for the tasks to schedule each task with a location on a queue.
% 

Modeling the elements of the monitoring platform is necessary to be able to study certain properties of the system.
The most important goal of a monitoring platform is to enable the autonomous operation of a set of services according to their SLA.
Thus, it is essential how to analyze that the monitoring platform can provide certain guarantees about the service and its SLA.
In addition, it is important be able to verify the monitoring platform through model checking and schedulability analysis.
Using timed automata and task automata facilitates model checking and verification through formal method tools such as UPPAAL~\cite{uppaal2004} supporting advanced methods such as state-space reduction~\cite{larsen1997efficient}. 
   
We use task automata as defined in \cite{fersman2007task,jaghoori2012composing,jaghoori2010time}.
Task automata are an extension of timed automata~\cite{alur:1994:timedautomata}.
In addition, we design the automata for the monitoring platform using the real-time extension of task automata presented in \cite{jaghoori2010time} p. 92 in which the author presents a mapping from Real-Time ABS~\cite{johnsen2012modeling} to the equivalent task automata.

A task type is a piece of executable program/code represented by a tuple $(b,w,d)$, where $b$ and $w$ respectively are the best-case and worst-case execution times and $d$ is the deadline.
In a task automata, there are two types of transitions: \emph{delay} and \emph{discrete}.
A delay transition models the execution of a running task by idling for other tasks.
A discrete transition corresponds to the arrival of a new task.
When a new task is triggered, it is placed into a certain position in the queue based on a scheduling policy~\cite{Nobakht:sched,nobakht2013future}.
Examples of a scheduling policy are \textsf{FIFO} or \textsf{EDF} (earliest deadline first).
The scheduling policy is modeled as a timed automaton \textsf{Sch}.
Every task has its own stop watch.
The scheduler also maintains a separate stop watch for each task to determine if a task misses its deadline.
All stop watches work at the same clock speed specified by $T$.

We design separate automata for each service $s$ characteristic: service availability $\alpha(s,\tau,t_c)$ by an automata $M_{\alpha_s}$ and service budget compliance $\beta(s,\tau)$, by an automata $M_{\beta_s}$.
Each automaton is responsible for one goal: to optimize the service characteristic.
$M_{\alpha_s}$ aims to improve $\alpha(s,\tau,t_c)$ whereas $M_{\beta_s}$ aims to improve $\beta(s,\tau)$.
$M_{\alpha_s}$ uses \jtt{allocate} to launch a new resource in the environment and improve the service $s$.
In contrast, $M_{\beta_s}$ uses \jtt{deallocate} to terminate a resource to decrease the cost of the service.

We use task automata to design $M_{\alpha_s}$.
Periodically, $M_{\alpha_s}$ checks whether the service availability
is within the thresholds, taking tolerance into account
(Definition~\ref{def:service:availability}).
If the condition fails, $M_{\alpha_s}$ generates a task for monitoring platform $P$ to allocate a new resource to service $s$ with a deadline of $\tau$.
We define the period to be $\tau$.
We use the semantics of a task automata in \cite{jaghoori2010time} p. 92 in the transitions of the task automata.
Figure~\ref{fig:fsm:task:alpha} and \ref{fig:fsm:task:beta} present $M_{\alpha_s}$ and $M_{\beta_s}$.
Both $M_{\alpha_s}$ and $M_{\beta_s}$ share state with the monitoring platform $P$.
The state keeps the current number of resources for a service $s$ that is denoted by $\sigma(s)$.
All timed automata and task automata in the monitoring platform have shared access to $\sigma(s)$.
In the automata, we use a conditional statement to check the service characteristics $\alpha(s,\tau,t_c)$ or $\beta(s,\tau)$.
If the condition fails, $M_{\alpha_s}$ requests $P$ to \jtt{allocate} a new resource to $s$ and $M_{\beta_s}$ requests $P$ to \jtt{deallocate} a resource.
In addition, $M_{\alpha_s}$ triggers a new task $\jtt{verify}_\alpha$ with deadline $t_G$.
Intuitively, this means the service characteristic $\alpha(s,\tau,t_c)$ is verified to be within the expected thresholds after at most $t_G$ time.
\begin{figure}[h]
% \vspace{-15pt}
\captionsetup[subfigure]{font=footnotesize}
\centering
\subcaptionbox{\scriptsize{$M_{\alpha_s}$ task automata for $\alpha(s,\tau,t_c)$}\label{fig:fsm:task:alpha}}[0.45\textwidth]{
\begin{tikzpicture}[->,>=stealth',shorten >=1pt,auto,node distance=7cm,thick,scale=0.6,every node/.style={scale=0.6},level distance=20mm]
\node[state,initial]
  (a)   {};
\node[state]
  (b) [right of=a]  {};
\path[]
(a)
  edge
  node {$\bfjtt{duration}(\tau,\tau)$}
  (b)
(b)
  edge [bend right,above] 
  node [align=center] 
    {$\bfjtt{if}\big((1-\varepsilon_\alpha(s,\tau,t_c)) > \alpha(s,\tau,t_c)\big) $ \\ \{ %\\
     $P \; ! \; \jtt{allocate}(s,\tau)$ ; %\\
     $P \; ! \; \jtt{verify}_\alpha(s,t_G)$ \}
    }
  (a)
;
\end{tikzpicture}
}%
\subcaptionbox{\scriptsize{$M_{\beta_s}$ task automata for $\beta(s,\tau)$}\label{fig:fsm:task:beta}}[0.45\textwidth]{
\begin{tikzpicture}[->,>=stealth',shorten >=1pt,auto,node distance=7cm,thick,scale=0.6,every node/.style={scale=0.6},level distance=20mm]
\node[state,initial]
  (a)   {};
\node[state]
  (b) [right of=a]  {};
\path[]
(a)
  edge 
  node {$\bfjtt{duration}(\tau,\tau)$}
  (b)
(b)
  edge [bend right, above] 
  node [align=center] 
       {$\bfjtt{if}\big((1-\varepsilon_\beta(s,\tau)) > \beta(s,\tau)\big) $ 
       \\ \{ %\\
        $P \; ! \; \jtt{deallocate}(s,\tau)$ ; %\\
        $P \; ! \; \jtt{verify}_\beta(s,t_G)$ \}
       }
  (a)
;
\end{tikzpicture}
}
% \vspace{-10pt}
% \caption{{\scriptsize Task Automata for $\alpha(s,\tau,t_c)$ and $\beta(s,\tau)$}}
% \vspace{-10pt}
\end{figure}

We use a separate task automaton for each service characteristic to verify the SLA of the service after $t_G$ time.
Respectively, $M^{\alpha}_{\jtt{V}}$ and $M^{\beta}_{\jtt{V}}$ execute tasks $\jtt{verify}_\alpha$ and $\jtt{verify}_\beta$ (Figures~\ref{fig:fsm:task:verify:alpha} and \ref{fig:fsm:task:verify:beta}).
$M^{\alpha}_{\jtt{V}}$ uses \bfjtt{await} to ensure the condition of the SLA.
In addition, the task is controlled by the scheduler using a deadline that is specified as $t_G$ in the generated task $\jtt{verify}_\alpha(s,t_G)$ in $M_{\alpha_s}$. 
If $t_G$ passes before the guard statement in \bfjtt{await} statement holds, it leads to a \emph{missed deadline}.
\begin{figure}[h]
% \vspace{-20pt}
\captionsetup[subfigure]{font=scriptsize}
\centering
\subcaptionbox{\scriptsize{$M^{\alpha}_{\jtt{V}}$ to execute $\jtt{verify}_\alpha$}\label{fig:fsm:task:verify:alpha}}[0.45\textwidth]{
\begin{tikzpicture}[->,>=stealth',shorten >=1pt,auto,node distance=6.5cm,thick,scale=0.6,every node/.style={scale=0.6},level distance=20mm]
\node[state,initial]
  (a)   {};
\node[state,accepting]
  (b) [right of=a]  {};
\path[]
(a)
  edge
  node [align=center] 
    {$\bfjtt{await} \; \alpha(s,\tau,t_c) \geq 1-\varepsilon_\alpha(s,\tau,t_c)$}
  (b)
;
\end{tikzpicture}
}%
\subcaptionbox{\scriptsize{$M^{\beta}_{\jtt{V}}$ to execute $\jtt{verify}_\beta$}\label{fig:fsm:task:verify:beta}}[0.45\textwidth]{
\begin{tikzpicture}[->,>=stealth',shorten >=1pt,auto,node distance=6cm,thick,scale=0.6,every node/.style={scale=0.6},level distance=20mm]
\node[state,initial]
  (a)   {};
\node[state,accepting]
  (b) [right of=a]  {};
\path[]
(a)
  edge
  node [align=center] 
    {$\bfjtt{await} \; \beta(s,\tau) \geq 1-\varepsilon_\beta(s,\tau)$}
  (b)
;
\end{tikzpicture}%
}
% \vspace{-10pt}
% \caption{{\scriptsize Task automata to execute $\jtt{verify}_\alpha$ and $\jtt{verify}_\beta$}}
% \vspace{-10pt}
\end{figure}

Both $M_{\alpha_s}$ and $M_{\beta_s}$ are specific to one particular service $s$.
A generalized automaton for all services is obtained as their parallel composition:$M_\alpha = (\parallel_s M_{\alpha_s})$ and $M_\beta = (\parallel_s M_{\beta_s})$.
The tasks generated by $M_\alpha$ and $M_\beta$ (triggered by the calls to
\jtt{allocate} and \jtt{deallocate}) are executed by the task automata for platform $M_P$.

We model monitoring platform $P$ by a task automata $M_P$.
The task types are $\{\jtt{A}(\jtt{allocate}),\allowbreak \jtt{D}(\jtt{deallocate})\}$.
For task type \jtt{A} in $M_P$, we use $(b,w,d) = (t_i,\tau,\tau)$; i.e.
the best-case execution time of a task is the resource initialization time, 
the worst-case is the length of the monitoring window, and
the deadline is the length of the monitoring window.
For task type \jtt{D} in $M_P$, we use $(b,w,d) =(0, \tau,\tau)$. 
We do not fix the scheduling policy \textsf{Sch}.
The error state $q_{err}$ in $M_P$ is defined when either a deadline is missed or when the platform fails to provision a resource.
Thus the monitoring platform $P$ contains the following ingredients:
\[
M_P = \langle M_{\jtt{A}} \parallel M_{\jtt{D}} \parallel 
M^{\alpha}_{\jtt{V}} \parallel M^{\beta}_{\jtt{V}}
, \textsf{Sch}, \tau \rangle
\]
We define $M_{\jtt{A}_s}$ as the timed automata to execute the tasks of type \jtt{allocate} in $M_P$.
We use the model  semantics presented in \cite{jaghoori2010time} p. 92 to design $M_{\jtt{A}_s}$. The resulting automata is presented in Figure~\ref{fig:fsm:alloc_s}.

\begin{figure}[h]
% \vspace{-20pt}
\centering
\begin{tikzpicture}[->,>=stealth',shorten >=1pt,auto,node distance=4cm,thick,scale=0.7,every node/.style={scale=0.7}]
\node[state,initial]    
  (a)   {};
\node[state]
  (b) [right of=a]   {};
\node[state,accepting]
  (c) [right of=b]  {};
\path[]
(a) 
    edge    
    node {$\bfjtt{duration}(t_i,\tau)$}          
    (b)
(b) 
    edge    
    node {$\sigma(s) \gets \sigma(s) + 1$} 
    (c)
;
\end{tikzpicture}
% \vspace{-15pt}
\caption{{\scriptsize{$M_{\jtt{A}_s}$: Timed Automaton to execute task type \jtt{allocate} in $M_P$}}
% \vspace{-20pt}
\label{fig:fsm:alloc_s}}
\end{figure}

Then, we define $M_{\jtt{A}}$ in $M_P$ as:
$M_{\jtt{A}} = \; \parallel_s M_{\jtt{A}_s}$; i.e.
the composition of all timed automata to execute a task \jtt{allocate} for some service $s$.
% 
Similarly, we design $M_{\jtt{D}_s}$ to execute task type \jtt{deallocate} in Figure~\ref{fig:fsm:dealloc_s}.
Therefore, we also have $M_{\jtt{D}}$ in $M_P$ as: $M_{\jtt{D}} = \; \parallel_s M_{\jtt{D}_s}$.
% 
\begin{figure}[h!]
% \vspace{-10pt}
\centering
\begin{tikzpicture}[->,>=stealth',shorten >=1pt,auto,node distance=4cm,thick,scale=0.7,every node/.style={scale=0.7}]
\node[state,initial]    
  (a)   {};
\node[state]
  (b) [right of=a]   {};
\node[state,accepting]
  (c) [right of=b]  {};
\path[]
(a) 
    edge    
    node {$\bfjtt{duration}(t_i,\tau)$}          
    (b)
(b) 
    edge    
    node {$\sigma(s) \gets \sigma(s) - 1$} 
    (c)
;
\end{tikzpicture}
% \vspace{-15pt}
\caption{{\scriptsize{$M_{\jtt{D}_s}$: Timed Automaton to execute task type \jtt{deallocate} in $M_P$}}
% \vspace{-20pt}
\label{fig:fsm:dealloc_s}}
\end{figure}
% 

For a particular service $s$, its automaton $M_{\alpha_s}$ regularly measures the service
characteristics and calculates $\alpha(s,\tau,t_c)$. When $s$ is under-capacity, $M_{\alpha_s}$ requests to \jtt{allocate} a new resource for $s$ through monitoring platform $P$.
This generates a new task in $M_P$ that is executed by $M_{\jtt{A}_s}$.
When the task completes, the state of the service $\sigma(s)$ is updated; strictly increased.
Thus, in isolation, the combination of $M_{\alpha_s}$ and $M_{\jtt{A}_s}$ increase the value of service availability $\alpha(s,\tau,t_c)$ for service $s$ over time.
Similarly, in isolation, the combination of $M_{\beta_s}$ and $M_{\jtt{D}_s}$ increase the value of service budget compliance $\beta(s,\tau)$ for service $s$ over time. 
Because in the latter, \jtt{deallocate} is used to decrease the cost of the service and as such increases $\beta(s,\tau)$. 

In reality, resources might fail in the environment.
The failure of a resource is not and cannot be controlled by the monitoring platform $P$.
However, the failure of a resource affects the state of a service and its characteristics.
Thus, we model the environment, including failures, as an additional timed automata, $M_E$.
In $M_E$, in every monitoring window, there is a probability that some resources fail.
For example, we present a particular instance of $M_E$ in Figure~\ref{fsm:env}.
In this environment, in every monitoring, an unspecified constant ($c$) number of resources fail.

\begin{figure}[h]
% \vspace{-20pt}
\centering
\begin{tikzpicture}[->,>=stealth',shorten >=1pt,auto,node distance=5cm,thick,scale=0.7,every node/.style={scale=0.7}]
\node[state,initial]
  (a) {};
\node[state]
  (b) [right of=a] {};
\node[state]
  (c) [right of=b] {};
\path[]
(a)
  edge
  node {$\bfjtt{duration}(0,\tau)$}
  (b)
(b)
  edge
  node {$\sigma(s)\gets\sigma(s)-c$}
  (c)
;
\end{tikzpicture}
% \vspace{-13pt}
\caption{\scriptsize{An example behavior for $M_E$}}
\label{fsm:env}
% \vspace{-20pt}
\end{figure}

We define system automata~\cite{jaghoori2010time} (p. 33, Definition 3.2.7) for each service characteristic;
$\mathcal{S}_\alpha$ for $\alpha(s,\tau,t_c)$ and $\mathcal{S}_\beta$ for $\beta(s,\tau)$:
\[
\mathcal{S}_\alpha = M_\alpha \parallel M_E \parallel M_P
\quad\quad\textit{ and }\quad\quad
\mathcal{S}_\beta = M_\beta \parallel M_E \parallel M_P
\]

With the above automata that we designed for $\alpha(s,\tau,t_c)$ and $\beta(s,\tau)$, we are now ready to present the main results.
% 

\begin{thm}
\label{thm:alpha}
If the SLA for service $s$ on $\alpha(s,\tau,t_c)$ is violated, either:
\begin{itemize}
\item $\mathcal{S}_\alpha$ re-establishes the condition $\alpha(s,\tau,t_c) \geq 1 - \varepsilon_\alpha(s,\tau)$ (thereby satisfying the SLA) within $t_G$ time, or, 
\item there exists at least one task $\jtt{verify}_\alpha$ in $M^{\alpha}_{\jtt{V}}$ with a missed deadline.
\end{itemize}
\end{thm}

\begin{proof}
At any given time in $T$:
\begin{itemize}
\item If $\alpha(s,\tau,t_c) \geq 1-\varepsilon_\alpha(s,\tau)$, then the SLA for service availability $\alpha$ is satisfied. 
\item If the above condition does not hold, on every monitoring window $\tau$, $M_\alpha$ generates a new task \jtt{allocate} in $M_{\jtt{A}}$. 
In addition, a new task $\jtt{verify}_{\alpha}$ is generated with a deadline $t_G$. 
After a duration of $t_G$, the \bfjtt{await} statement allows $M^{\alpha}_{\jtt{V}}$ to complete the task $\jtt{verify}_\alpha$ only if the condition $\alpha(s,\tau,t_c) \geq 1-\varepsilon_\alpha(s,\tau)$ holds.
If this is not the case, since $t_G$ has passed, the scheduler generates a missed deadline (moving to its error state). 
\end{itemize}
\end{proof}

\begin{thm}
\label{thm:beta}
If the SLA for service $s$ on $\beta(s,\tau)$ is violated, either:
\begin{itemize}
\item $\mathcal{S}_\beta$ re-establishes the condition $\beta(s,\tau) \geq 1-\varepsilon_\beta(s,\tau)$ (thereby satisfying the SLA) within $t_G$ time, or,
\item there exists at least one task $\jtt{verify}_\beta$ in $M^{\beta}_{\jtt{V}}$ with a missed deadline. 
\end{itemize}
\end{thm}

\begin{proof}
Similar to the proof of Theorem~\ref{thm:alpha}.
\end{proof}

In practice, the guarantee of $\mathcal{S}_\alpha$ and $\mathcal{S}_\beta$
in isolation to eventually evolve the system to satisfy the SLA is not enough.
In reality, a service provider tries ensure both simultaneously to reduce their cost of service delivery while ensuring the delivered service is of the expectations agreed upon with the customer.
However, these goals conflict.
When $\alpha(s,\tau,t_c)$ increases because of adding a new resource,
it means that service $s$ costs more, hence $\beta(s,\tau)$ decreases.
%Under the same $B(s,\tau)$, this means that $\beta(s,\tau)$ decreases.
The same applies in the other direction: increasing $\beta(s,\tau)$
negatively affects $\alpha(s,\tau,t_c)$.

To capture the combined behavior of service availability and budget compliance,
we compose them. 
We define \emph{service sustainability} $\gamma(s,\tau)$ as the composition of $\alpha(s,\tau,t_c)$ and $\beta(s,\tau)$.
We present the composition by system automata $\mathcal{S}_\gamma$ as:
\[
\mathcal{S}_\gamma = \mathcal{S}_\alpha \; \parallel \; \mathcal{S}_\beta
\]

Authors in~\cite{fersman2007task} define that a task automata is \emph{schedulable} if there exists no task on the queue that misses its deadline. 
The next theorem presents the relationship between schedulability analysis of service sustainability and satisfying its SLA.

\begin{thm}
\label{thm:gamma}
If $\mathcal{S}_\gamma$ is \emph{schedulable} given input parameters $(\tau, t_i, t_G)$, then the SLA for both service characteristics $\alpha(s,\tau,t_c)$ and $\beta(s,\tau)$ is satisfied within $t_G$ time after a violation.
\end{thm}

\begin{proof}
When a violation of the SLA occurs in $\mathcal{S}_\gamma$, either $\mathcal{S}_\alpha$ or $\mathcal{S}_\beta$ (or both) start to evolve the service based on Theorems~\ref{thm:alpha} and \ref{thm:beta}.
Therefore, there exists at least one task of $\jtt{verify}_\alpha$ or $\jtt{verify}_\beta$ with a deadline $t_G$.
% 
Hence, if $\mathcal{S}_\gamma$ is schedulable, then neither $\jtt{verify}_\alpha$ nor $\jtt{verify}_\beta$ miss their deadline.
Thus, both $\mathcal{S}_\alpha$ and $\mathcal{S}_\beta$ are schedulable.
This means that both $\jtt{verify}_\alpha$ and $\jtt{verify}_\beta$ complete successfully.
Therefore, the SLA of the service is guaranteed within $t_G$ after a violation in $\mathcal{S}_\gamma$.
\end{proof}

Using the algorithm presented in Chapter 6~\cite{jaghoori2010time}, we translate the above task automata into traditional timed automata.
This allows to leverage well-established model checking techniques such as UPPAAL~\cite{uppaal2004} to determine the schedulability of $\mathcal{S}_\gamma$.
Moreover, the results of the schedulability analysis serves as a method to optimize the input parameters of the monitoring model including $\tau$ and $t_G$.